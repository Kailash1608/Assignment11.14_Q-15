% \iffalse
\let\negmedspace\undefined
\let\negthickspace\undefined
\documentclass[journal,12pt,twocolumn]{IEEEtran}
\usepackage{cite}
\usepackage{amsmath,amssymb,amsfonts,amsthm}
\usepackage{algorithmic}
\usepackage{graphicx}
\usepackage{textcomp}
\usepackage{xcolor}
\usepackage{txfonts}
\usepackage{listings}
\usepackage{enumitem}
\usepackage{mathtools}
\usepackage{gensymb}
\usepackage{comment}
\usepackage[breaklinks=true]{hyperref}
\usepackage{tkz-euclide} 
\usepackage{listings}
\usepackage{gvv}
\def\inputGnumericTable{}
\usepackage[latin1]{inputenc}                              
\usepackage{color}                                            
\usepackage{array}                                            
\usepackage{longtable}                                       
\usepackage{calc}                                             
\usepackage{multirow}                                         
\usepackage{hhline}                                           
\usepackage{ifthen}                                           
\usepackage{lscape}

\newtheorem{theorem}{Theorem}[section]
\newtheorem{problem}{Problem}
\newtheorem{proposition}{Proposition}[section]
\newtheorem{lemma}{Lemma}[section]
\newtheorem{corollary}[theorem]{Corollary}
\newtheorem{example}{Example}[section]
\newtheorem{definition}[problem]{Definition}
\newcommand{\BEQA}{\begin{eqnarray}}
\newcommand{\EEQA}{\end{eqnarray}}
\newcommand{\define}{\stackrel{\triangle}{=}}
\theoremstyle{remark}
\newtheorem{rem}{Remark}
\begin{document}

\bibliographystyle{IEEEtran}
\vspace{3cm}

\title{NCERT 11.14 Q-25}
\author{EE23BTECH11207 -KAILASH.C$^{}$% <-this % stops a space
}
\maketitle
\newpage
\bigskip

\renewcommand{\thefigure}{\theenumi}
\renewcommand{\thetable}{\theenumi}
\section{\textit{QUESTION:}}
\Large
{A mass attached to a spring is free to oscillate, with angular velocity $\omega$, in a horizontal
plane without friction or damping. It is pulled to a distance $x_0$
 and pushed towards
the centre with a velocity $v_0$
 at time t = 0. Determine the amplitude of the resulting
oscillations in terms of the parameters $\omega{}$, $x_0$,
 and $v_0$
. [Hint : Start with the equation
$x = a \cos(\omega{t}+\theta{})$ and note that the initial velocity is negative.]}
\section{\textit{SOLUTION:}}
\begin{table}[htbp]
    \centering
    \def\arraystretch{1.5}
    \begin{tabular}{|p{2cm}|p{2cm}|}
    \hline
    Symbols&Definition \\  \hline
    x&Displacement \\  \hline
    t&Time \\ \hline
    v&Velocity \\ \hline
    $\omega$&Angular velocity \\  \hline
    $\Theta$&Phase constant \\  \hline
    \end{tabular}
    \caption{Symbols and their definitions}
    \label{tab:Table1}
\end{table}    
\begin{align}    
x=A\cos(\omega{t}+\theta)\end{align}
By differentiating equation-(1) with respect to time t,
we get:
\begin{align}
v&=\frac{dx}{dt}\\
v&=-A\omega{}\sin(\omega{t}+\theta{})  
\end{align}
Let at time t=0s, x=$x_0$\\
Substituting t=0s in equation (1):
\begin{align}
x&=A\cos(\omega{(0)}+\theta{})\\
x_0&=A\cos(\theta{})
\end{align}
\begin{align}
v_0&=\frac{dx_0}{dt}\\
v_0&=-A\omega{}\sin(\omega{(0)}+\theta{})\\
v_0&=-A\omega{}\sin(\theta{})\\
-A\sin(\theta{})&=\frac{v_0}{\omega{}}
\end{align}
Squaring and Adding eq-(5) and eq-(9):\\
$(A\cos(\theta{}))^2+(A\sin(\theta{})^2)$
\begin{align}
=&(x_0)^2+(\frac{v_0}{\omega{}})^2\end{align}
$A^2(\cos^2(\theta{})+\sin^2(\theta{}))$
\begin{align}
&=(x_0)^2+(\frac{v_o}{\omega{}})^2\end{align}
\begin{align}
A^2&=(x_0)^2+(\frac{v_o}{\omega{}})^2\end{align}
Therefore by solving the above equation we get:\\
\LARGE{Amplitude=A=$\sqrt{(x_0)^2+(\frac{v_0}{\omega{}})^2}$}
\end{document}

